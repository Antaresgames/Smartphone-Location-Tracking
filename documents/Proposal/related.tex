\section{Related Work}
\label{sec:related}
The goal of the paper by Jackson et. al.\cite{Jackson:2006:LTT:1376148.1376210} was similar to ours - to keep track of an object in an area using accelerometers. Useful information such as calibration, calculation and data retrieval were relevant topics discussed.

The goal of the paper by Liu et. al.\cite{Liu:2012:PLW:2348543.2348581} was to use multiple cell phone users in conjunction with Wi-Fi networks to help limit the error of a Wi-Fi only solution. This paper details the multiple deficiencies of Wi-Fi localization and will be useful when we try to calibrate for them.

The paper by Biswas et. al.\cite{Biswas_2010_6819} uses Wi-Fi localization to direct a robot around a building while avoiding static and dynamic obstacles. This relates to our project because we will also need to determine location based partially off of Wi-Fi localization.

The paper by Reuveny et. al.\cite{unpublished-minimal} was about a project that used an accelerometer and 2 gyroscopes in order to continually keep track of motion. This paper contains information about accelerometer and gyroscope properties and how to implement the tracking algorithm. This information is useful because we will need to be doing similar activities for our project.

